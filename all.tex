\documentclass[12pt]{article}
\usepackage{amsmath,amssymb,amsthm,amsfonts}
\usepackage{mathtools}
\usepackage{enumitem}
\usepackage{hyperref}
\usepackage{cleveref}
\usepackage[margin=1in]{geometry}
\usepackage[T1]{fontenc}
\usepackage[utf8]{inputenc}
\usepackage{xcolor}
\usepackage{color}
\usepackage{listings}
\usepackage{tabularx}

\hypersetup{
    colorlinks=true,
    linkcolor=blue,
    filecolor=magenta,      
    urlcolor=cyan,
    pdftitle={Residue Environment Analysis for Collatz-like Dynamics},
    pdfpagemode=FullScreen,
}

\theoremstyle{plain}
\newtheorem{theorem}{Theorem}[section]
\newtheorem{lemma}[theorem]{Lemma}
\newtheorem{proposition}[theorem]{Proposition}
\newtheorem{corollary}[theorem]{Corollary}
\newtheorem{conjecture}[theorem]{Conjecture}
\newtheorem{verification}[theorem]{Verification}

\theoremstyle{definition}
\newtheorem{definition}[theorem]{Definition}
\newtheorem{example}[theorem]{Example}
\newtheorem{remark}[theorem]{Remark}

\DeclareMathOperator{\lcm}{lcm}
\DeclareMathOperator{\li}{li}
\DeclareMathOperator{\Mod}{mod}
\DeclareMathOperator{\ord}{ord}
\DeclareMathOperator{\Res}{Res}
\DeclareMathOperator{\supp}{supp}
\DeclareMathOperator{\Vol}{Vol}

\newcommand{\Z}{\mathbb{Z}}
\newcommand{\R}{\mathbb{R}}
\newcommand{\Q}{\mathbb{Q}}
\newcommand{\N}{\mathbb{N}}
\newcommand{\C}{\mathbb{C}}
\newcommand{\PP}{\mathbb{P}}
\newcommand{\eps}{\varepsilon}
\newcommand{\vphi}{\varphi}
\newcommand{\ph}{\phi}

\lstdefinelanguage{lean}{
  keywords={
    def, theorem, lemma, proposition, corollary, example,
    by, enter, have, let, assume, show, calc, exact,
    if, then, else, in, class, instance, structure, inductive,
    rw, simp, intro, apply, unfold, exact, sorry, trivial,
    Prop, Type, Sort, Nat, ℕ, Int, ℤ, Fin, Odd,
    open, import, set_option
  },
  sensitive=true,
  comment=[l]{--},
  morestring=[b]",
  morestring=[s]'/
}


\definecolor{listingbg}{HTML}{F0F8FF}
\lstset{
  language=lean,
  inputencoding=utf8,
  extendedchars=true,
  basicstyle=\ttfamily\small,
  backgroundcolor=\color{listingbg},
  keywordstyle=\color{blue}\bfseries,
  commentstyle=\color{gray}\itshape,
  stringstyle=\color{teal},
  numbers=left,
  numberstyle=\tiny,
  stepnumber=1,
  numbersep=8pt,
  tabsize=2,
  keepspaces=true, 
  columns=flexible,
  breaklines=true,
  breakatwhitespace=true,
  showstringspaces=false,
  frame=single,
  captionpos=b,
  literate=
    % arrows & logical connectives
    {→}{{$\to$}}1
    {←}{{$\leftarrow$}}1
    {↔}{{$\leftrightarrow$}}1
    {↦}{{$\mapsto$}}1
    {∧}{{$\land$}}1
    {∨}{{$\lor$}}1
    {¬}{{$\neg$}}1
    % quantifiers & existence
    {∀}{{$\forall$}}1
    {∃}{{$\exists$}}1
    % relations
    {≤}{{$\le$}}1
    {≥}{{$\ge$}}1
    {≠}{{$\neq$}}1
    {≡}{{$\equiv$}}1
    {∣}{{$\mid$}}1
    % brackets for tuples/angle‐notation
    {⟨}{{$\langle$}}1
    {⟩}{{$\rangle$}}1
    % math sets & number‐systems
    {ℕ}{{$\mathbb{N}$}}1
    {ℤ}{{$\mathbb{Z}$}}1
    {ℚ}{{$\mathbb{Q}$}}1
    {ℝ}{{$\mathbb{R}$}}1
    {ℂ}{{$\mathbb{C}$}}1
    % tactic bullet
    {·}{{$\cdot$}}1
    % common type‐variable Greek letters
    {α}{{$\alpha$}}1
    {β}{{$\beta$}}1
    {γ}{{$\gamma$}}1
    {δ}{{$\delta$}}1
    {ε}{{$\varepsilon$}}1
    {ζ}{{$\zeta$}}1
    {η}{{$\eta$}}1
    {θ}{{$\theta$}}1
    {ι}{{$\iota$}}1
    {κ}{{$\kappa$}}1
    {λ}{{$\lambda$}}1
}

\title{Residue Environment Analysis for Collatz-like Dynamics}
\author{Stuart Riffle}
\date{\today}

\begin{document}
\maketitle
\begin{abstract}
\end{abstract}

\section{Introduction}
    \subsection{Definitions}





Odd core decomposition
Reference table 
Local arithmetic constraints



Information flow
- shifting is harmless
- role of LSB and MSB


Radical 

Natural order
- Definition
- Explain nested prime cycles
- Piano roll visualization
- Define radical, show it maps to factor sets
- Gaps are symmetric
- At least one per cycle when masking a bidirectional walk




Prime cycle gaps

Odd core environment
- Reference table

Odd core graph
- Diagram


\newpage


- The odd-to-odd Collatz map can be defined as a shift left in ternary, and a shift right in binary, with a +1 operation in between.
- Both shifts are information preserving, because they only shuffle zeroes. Pushing n around in this base or the other is reversible, so no information is lost.
- The binary avalanche caused by +1 is the only meaningful change of state in the Collatz map. The shifts are temporary and informationally harmless.
- The number of bits flipped in the avalanche is k.
- The per-step information evolution of Collatz is proportional to k. The nature of it is isomorphic to the rules of carry propagation.

- The binary carry avalanche maximizes diffusion of information from odd prime rings into the dyadic hierarchy (powers of two).
- Each bit flipped in the avalanche shuffles the corresponding mod 2^k class, erasing low-order memory.
- Information diffused by odd primes is not shed until the dyadic hierarchy is "drained", which happens whenever the avalanche rolls over the MSB.

Symbolic progression
- The LSB of n under the odd-to-odd map is always 1 (because it's odd). The must also be an MSB _somewhere_ (because we're still playing Collatz, so n > 1).
- These are delimiters without informational value. The interior "payload" bits completely determine the state of the process, which ends when the delimiters join at one.
- The MSB protects the leading zeroes in the payload. The LSB protects the trailing zeroes, and serves as a "stop" during the /2 normalization step. Together they form a frame.
- The interior bits as a string form a symbolic progression, isomorphic to the arithmetic by construction (you can always wrap the string in delimiters again to recover n).


- Integers have a finite MSB, so a carry avalanche is guaranteed to terminate, but 2-adic values are infinite bit strings, and sensitive to the low order bits instead. A pure 2-adic representation does not sieve radicals, so it can't make progress, and exhibits ergodic behavior. When the avalanche reaches the MSB, it is replaced by the carry, destroying information.

- (As an aside, this is why Minsky's two-tag machine traces Collatz: it's a cleverly obfuscated carry-propagator).



\begin{thebibliography}{9}

\end{thebibliography}

\end{document}

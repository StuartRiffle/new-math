\documentclass[12pt]{article}
\usepackage{amsmath,amssymb,amsthm,amsfonts}
\usepackage{mathtools}
\usepackage{enumitem}
\usepackage{hyperref}
\usepackage{cleveref}
\usepackage[margin=1in]{geometry}
\usepackage[T1]{fontenc}
\usepackage[utf8]{inputenc}
\usepackage{newunicodechar}
\usepackage{caption}
\usepackage{xcolor}
\usepackage{color}
\usepackage{listings}
\usepackage{tabularx}

\hypersetup{
    colorlinks=true,
    linkcolor=blue,
    filecolor=magenta,      
    urlcolor=cyan,
    pdftitle={Residue Environment Analysis for Collatz-like Dynamics},
    pdfpagemode=FullScreen,
}

\theoremstyle{plain}
\newtheorem{theorem}{Theorem}[section]
\newtheorem{lemma}[theorem]{Lemma}
\newtheorem{proposition}[theorem]{Proposition}
\newtheorem{corollary}[theorem]{Corollary}
\newtheorem{conjecture}[theorem]{Conjecture}
\newtheorem{verification}[theorem]{Verification}

\theoremstyle{definition}
\newtheorem{definition}[theorem]{Definition}
\newtheorem{example}[theorem]{Example}
\newtheorem{remark}[theorem]{Remark}

\DeclareMathOperator{\lcm}{lcm}
\DeclareMathOperator{\li}{li}
\DeclareMathOperator{\Mod}{mod}
\DeclareMathOperator{\ord}{ord}
\DeclareMathOperator{\Res}{Res}
\DeclareMathOperator{\supp}{supp}
\DeclareMathOperator{\Vol}{Vol}

\newcommand{\Z}{\mathbb{Z}}
\newcommand{\R}{\mathbb{R}}
\newcommand{\Q}{\mathbb{Q}}
\newcommand{\N}{\mathbb{N}}
\newcommand{\C}{\mathbb{C}}
\newcommand{\PP}{\mathbb{P}}
\newcommand{\eps}{\varepsilon}
\newcommand{\vphi}{\varphi}
\newcommand{\ph}{\phi}

\lstdefinelanguage{lean}{
  keywords={
    def, theorem, lemma, proposition, corollary, example,
    by, enter, have, let, assume, show, calc, exact,
    if, then, else, in, class, instance, structure, inductive,
    rw, simp, intro, apply, unfold, exact, sorry, trivial,
    Prop, Type, Sort, Nat, ℕ, Int, ℤ, Fin, Odd,
    open, import, set_option
  },
  sensitive=true,
  comment=[l]{--},
  morestring=[b]",
  morestring=[s]'/
}


\definecolor{listingbg}{HTML}{F0F8FF}
\lstset{
  language=lean,
  inputencoding=utf8,
  extendedchars=true,
  basicstyle=\ttfamily\small,
  backgroundcolor=\color{listingbg},
  keywordstyle=\color{blue}\bfseries,
  commentstyle=\color{gray}\itshape,
  stringstyle=\color{teal},
  numbers=left,
  numberstyle=\tiny,
  stepnumber=1,
  numbersep=8pt,
  tabsize=2,
  keepspaces=true, 
  columns=flexible,
  breaklines=true,
  breakatwhitespace=true,
  showstringspaces=false,
  frame=single,
  captionpos=b,
  literate=
    % arrows & logical connectives
    {→}{{$\to$}}1
    {←}{{$\leftarrow$}}1
    {↔}{{$\leftrightarrow$}}1
    {↦}{{$\mapsto$}}1
    {∧}{{$\land$}}1
    {∨}{{$\lor$}}1
    {¬}{{$\neg$}}1
    % quantifiers & existence
    {∀}{{$\forall$}}1
    {∃}{{$\exists$}}1
    % relations
    {≤}{{$\le$}}1
    {≥}{{$\ge$}}1
    {≠}{{$\neq$}}1
    {≡}{{$\equiv$}}1
    {∣}{{$\mid$}}1
    % brackets for tuples/angle‐notation
    {⟨}{{$\langle$}}1
    {⟩}{{$\rangle$}}1
    % math sets & number‐systems
    {ℕ}{{$\mathbb{N}$}}1
    {ℤ}{{$\mathbb{Z}$}}1
    {ℚ}{{$\mathbb{Q}$}}1
    {ℝ}{{$\mathbb{R}$}}1
    {ℂ}{{$\mathbb{C}$}}1
    % tactic bullet
    {·}{{$\cdot$}}1
    % common type‐variable Greek letters
    {α}{{$\alpha$}}1
    {β}{{$\beta$}}1
    {γ}{{$\gamma$}}1
    {δ}{{$\delta$}}1
    {ε}{{$\varepsilon$}}1
    {ζ}{{$\zeta$}}1
    {η}{{$\eta$}}1
    {θ}{{$\theta$}}1
    {ι}{{$\iota$}}1
    {κ}{{$\kappa$}}1
    {λ}{{$\lambda$}}1
}

\newunicodechar{░}{\texttt{\symbol{"2591}}}
\newunicodechar{▒}{\texttt{\symbol{"2592}}}
\newunicodechar{▓}{\texttt{\symbol{"2593}}}
\newunicodechar{│}{\texttt{\symbol{"2502}}}
\newunicodechar{·}{\texttt{\symbol{"00B7}}}
\lstdefinelanguage{asciiart}{
    alsoletter={░▒▓│·},
    basicstyle=\ttfamily\small,
    showstringspaces=false
}

\title{Residue Environment Analysis for Collatz-like Dynamics}
\author{Stuart Riffle}
\date{\today}

\begin{document}
\maketitle
\begin{abstract}
\end{abstract}

\section{Introduction}
    \subsection{Definitions}





## Odd core decomposition

- Define:
    oc(x)   2-adic "odd core" after removing factors of 2 (an odd number is its own odd core)
    k(X)    power of two scaling, x = oc(c) * 2^k(x) (always 0 for odd numbers)
- Explain motivation

[Reference table of oc/k values for even integers, one page, dense]

- Define:
  neighbors   the even numbers n-1 and n+1 adjacent to odd n.
- Linking odd numbers to the odd cores of their even neighbors forms a self-similar directed graph over odd integers, terminating at 1.

[Diagram of odd core graph to 31 in top half of page with explanation underneath]

- General topology
- Local arithmetic constraints

## Natural order of factor sets

- Explain nested prime cycles

[Visualization of prime cycle misalignment and convergence]

- Explain totient theorem, frequency of combinations
- Define radical, show uneven distribution of factor sets in progression, and radical happens only once at the end

[Visualization of long prime cycle with one radical]

- Constraints on succession
- Fundamental asymmetry

[Piano roll visualization, full page, left side, commentary in right column]

- Show that radicals map to factor sets
- Show odd core graph partitions integers by radical, unique walk to root

[Visualization of radical equivalence class chains, full page, graphviz]

## Collatz functional machinery

- The MSB and LSB protect the internal payload
- Rolling over the MSB means you have reached a power of two, terminating the sequence
- The LSB functions as a backstop for normalization
- The value of n is a pointer

## Information flow

- The odd-to-odd Collatz map can be defined as ternary/binary shifts and +1
- Shuffling zeroes in any base is reversible, so no information is lost.
- The binary avalanche caused by +1 is the only meaningful change of state.
- The number of bits flipped in the avalanche is k.


## Factor set cursor
- 3n and /2 only affect those factors, higher prime factors not added/removed

[Visualization showing zeroes in residue vectors unchanged under scaling by 2 or 3]

- Explain chaos inveriance
- Explain cursor metaphor
- Define: cursor primes, cursor radical
- Show that Collatz walks cursor radicals
- Explain that 3n and /2 shuffle the order of the walk snipping it into series of discontinuous single steps between radicals, but that it's a permutation on all rings and radical transitions cycle.

[Empirical data TBD]

## Coprimality

- Explain that +1 breaks alignment
- Explain how this enumerates cursor radicals

[Visualization of the orbit of 27 in Collatz order (top half of page) then sorted by piano roll (bottom half) to demonstrate that nontrivial factor sets don't repeat]

## The persistent state is the CRT

- Explain how the CRT keeps track long term despite constant 3n and /2
- Show the "scrambled" residues preserve misalignment

[Visualization of residue vector relationship?]

## Argument for termination by cursor walk

- Show convergence at only powers of two
- Argument for termination


## Carry propagation

- Each bit flipped in the avalanche shuffles the corresponding mod 2^k class, erasing low-order memory.
- The binary carry avalanche maximizes diffusion of information from odd prime rings into the dyadic hierarchy (powers of two).

##Dissection of orbit 27

[Symbolic binary/ternary visualization showing rectangles corresponding to up-runs]

- Show systematic elimination of radicals
- Trace constraints leading to convergence

## Ergodicity

- Integers have a finite MSB, so a carry avalanche is guaranteed to terminate, but 2-adic values are infinite bit strings, and sensitive to the low order bits instead. 
- A pure 2-adic representation does not sieve radicals, so it can't make progress, and exhibits ergodic behavior.

TODO: argue for termination through loss of entropy?

## Local environment fingerprint

- The arithmetic properties of n are completely determined by the 2-adic valuations (odd core and k exponent) of its two immediate even neighbors n-1 and n+1.
- Define fingerprint as this pair of 2-adic valuations (4-tuple of integers).
- This fingerprint uniquely identifies n and its position in the odd core graph.
- List some invariants/relationships

## Odd core graph isomorphic to Collatz evolution

[Stopping distance group reference list, dense, one page]

[Odd core graph labeled by stopping distance]

- Note ping-pong between sides
- Note chains of repeated stopping times

[Case study: distance group 42 connectivity]

- Show that linked chains share a radical
- Explain why it's a stopping time equivalence class

[Case study: astronomical values]

## Automaton implementation

- The fingerprint of even n must have k-values (2-adic valuations) that follow one of two patterns: (>1, 1, >1, 1) or (1, >1, 1, >1).
- For odd n, it will be one of (>1, 1, 1, >1) or (1, >1, >1, 1).
- The Collatz process can run directly fingerprints; you don't need to use n at all.

[Automaton implementation, one page]

- Explain updating neighbor k without re-valuation
- Explain predicting avalanche using neighbor low bits

[Empirical data: automaton transition statistics]

- Argument for termination by automaton

## Symbolic analysis

- Explain payload extraction
- Explain isomorphism to arithmetic

[Case study: trace one odd-to-odd step of Minsky's two-tag machine, show that it's an unrolled carry-propagator isomorphic to the Collatz arithmetic]






- Count the number of consecutive low 1 bits of n, and subtract one. That is how many increasing odd-to-odd steps you are about to take. If there are 5 bits set, you can see 5 iterations into the future. 



## Symbolic analysis



## Dissection of orbit 27









Information flow
- shifting is harmless
- role of LSB and MSB


Radical 

Natural order







Odd core environment
- Reference table

Odd core graph
- Diagram


Prime cycle gaps
- Gaps are symmetric
- At least one per cycle when masking a bidirectional walk




\newpage






Symbolic progression
- The interior bits as a string form a symbolic progression, isomorphic to the arithmetic by construction (you can always wrap the string in delimiters again to recover n).



The mechanism of coprimality
- The product of n's unique prime factors is its radical. The value of n is always a multiple of its radical; it is aligned with it. Offset from n on both sides by the its radical is another number which is a multiple of the radical.
- Every combination of factors represents a radical, if you multiply them all together. There is a 1:1 correspondence between factor sets and radicals, because prime numbers are all coprime to each other, so every uniqe combination of prime factors has a unique product.
- Every radical is coprime to all other radicals, because primes are coprime to each other, so all their multiples are too.
- Prime factor set <==> prime radical, same thing.

Case study: walkthrough of n eliminating a radical:
- Show that 3n doesn't change the cursor
- Show that +1 breaks alignment with that radical
- Calculate cycle length, show 3n has no chance to return before then
- Explain the entire class of radicals is unreachable
- 
- Show that /2 does not destroy odd coprimality
- Explain accumulation of diffused odd prime information in the dyadic hierarchy

- Define the "natural" order of prime factor sets to be the one arising as you simply increment integers, the successor operation: {}, {2}, {3}, {2}, {5}, {2,3}, {7}, {2}, {3}, {2,5}, and so on.

\begin{lstlisting}
         base2        sym2 sym3         base3 dist    n                                               
         11011         ▓░▓ ▒▒▒│          1000   41    27                                           ▓░▓▒▒▒│                                
        101001        ░▓░░ █│││          1112   40    41                                         ░▓░░█│││                                   
         11111         ▓▓▓ ││▒│          1011   39    31                                         ▓▓▓││▒│       ▓▓▓│                               
        101111        ░▓▓▓ █▒█│          1202   38    47                                       ░▓▓▓█▒█│        ▓▓▓█                               
       1000111       ░░░▓▓ █│██          2122   37    71                                     ░░░▓▓█│██         ▓▓█│                               
       1101011       ▓░▓░▓ ███▒│        10222   36   107                                    ▓░▓░▓███▒│         ▓███                                   
      10100001      ░▓░░░░ ████│        12222   35   161                                  ░▓░░░░████│          ████                                   
       1111001       ▓▓▓░░ │││││        11111   34   121                                  ▓▓▓░░│││││          │││││                                   
       1011011       ░▓▓░▓ │▒│▒│        10101   33    91                                 ░▓▓░▓│▒│▒│                                                       
      10001001      ░░░▓░░ █▒▒█│        12002   32   137                               ░░░▓░░█▒▒█│                                                        
       1100111       ▓░░▓▓ ││█▒│        10211   31   103                               ▓░░▓▓│││█▒│          ▓▓                            
      10011011      ░░▓▓░▓ █▒██│        12202   30   155                             ░░▓▓░▓█▒██│            ▓█                            
      11101001      ▓▓░▓░░ ██│██        22122   29   233                            ▓▓░▓░░██│██             ██                            
      10101111      ░▓░▓▓▓ │││▒█        20111   28   175                           ░▓░▓▓▓│││▒█          ▓▓▓│││                             
     100000111     ░░░░░▓▓ █▒█▒▒│      100202   27   263                         ░░░░░▓▓█▒█▒▒│          ▓▓█                                 
     110001011     ▓░░░▓░▓ ██│█││      112122   26   395                        ▓░░░▓░▓██│█││           ▓██                                 
    1001010001    ░░▓░▓░░░ ███▒│█      210222   25   593                      ░░▓░▓░░░███▒│█            ███                                 
     110111101     ▓░▓▓▓▓░ ││││█│      121111   24   445                      ▓░▓▓▓▓░││││█│            ││││                                 
      10100111      ░▓░░▓▓ █│▒▒█        20012   23   167                      ░▓░░▓▓█│▒▒█           ▓▓█                                       
      11111011      ▓▓▓▓░▓ ██▒▒▒│      100022   22   251                     ▓▓▓▓░▓██▒▒▒│           ▓██                                       
     101111001     ░▓▓▓▓░░ ███│││      111222   21   377                   ░▓▓▓▓░░███│││ ───────    ███                                   
     100011011     ░░░▓▓░▓ ││││▒│      101111   20   283                  ░░░▓▓░▓││││▒│            ││││         (binary)    
     110101001     ▓░▓░▓░░ █▒█▒█│      120202   19   425                 ▓░▓░▓░░█▒█▒█│                             1                  
     100111111     ░░▓▓▓▓▓ ││██▒│      102211   18   319                ░░▓▓▓▓▓││██▒│ ─────    ▓▓▓▓▓ ────────    ▓▓▓▓▓      
     111011111     ▓▓░▓▓▓▓ █▒███│      122202   17   479               ▓▓░▓▓▓▓█▒███│           ▓▓▓▓█             ▓▓▓▓ █     
    1011001111    ░▓▓░░▓▓▓ ██│███      222122   16   719             ░▓▓░░▓▓▓██│███            ▓▓▓██             ▓▓▓ ██     
   10000110111   ░░░░▓▓░▓▓ ███▒│││    1110222   15  1079           ░░░░▓▓░▓▓███▒│││            ▓▓███             ▓▓ ███     
   11001010011   ▓░░▓░▓░░▓ ████│▒█    2012222   14  1619          ▓░░▓░▓░░▓████│▒█             ▓████             ▓ ████     
  100101111101  ░░▓░▓▓▓▓▓░ █████▒▒│  10022222   13  2429        ░░▓░▓▓▓▓▓░█████▒▒│ ─────       █████ ────────     █████     
    1110001111    ▓▓░░░▓▓▓ █▒█▒█▒│    1020202   12   911         ▓▓░░░▓▓▓█▒█▒█▒│ ─────── ─ ▓▓▓█                     2       
   10101010111   ░▓░▓░▓░▓▓ ██│█│█│    1212122   11  1367       ░▓░▓░▓░▓▓██│█│█│            ▓▓██                 (ternary)   
  100000000011  ░░░░░░░░░▓ ███▒│██    2210222   10  2051     ░░░░░░░░░▓███▒│██             ▓███               
  110000000101  ▓░░░░░░░▓░ ████│▒││  11012222    9  3077    ▓░░░░░░░▓░████│▒││ ───────── ─ ████            
    1001000001    ░░▓░░░░░ │▒│▒│█      210101    8   577     ░░▓░░░░░│▒│▒│█               │▒│▒│            
     110110001     ▓░▓▓░░░ │▒▒│█│      121001    7   433     ▓░▓▓░░░│▒▒│█│                                      
     101000101     ░▓░░░▓░ │▒▒▒││      110001    6   325    ░▓░░░▓░│▒▒▒││                                                 
        111101        ▓▓▓░ │█▒█          2021    5    61      ▓▓▓░│█▒█                                     
         10111         ░▓▓ █│█▒          0212    4    23      ░▓▓█│█▒                                   ▓▓█
        100011        ░░░▓ ██▒│          1022    3    35    ░░░▓██▒│                                    ▓██
        110101        ▓░▓░ ███│          1222    2    53   ▓░▓░███│                                     ███
           101           ░ █│              12    1     5     ░█│                                                    
             1             │                1    -     1      │                                                              
\end{lstlisting}

▓▓▓▓▓






- Natural ordering is a constrained mapping: consecutive integers share no factors.
- The +1 operation is a permutation on all prime integer rings, at once.
- The sequence is cyclical modulo the product of primes.

- 3n can only add a factor of 3, the rest of the factor set is unchanged.
- The same is true of the /2 step: scaling by powers of two can't add or remove odd prime factors.
- The set of prime factors above 3 is therefore chaos-invariant, and we can reason about its long term evolution.

- The +1 operation breaks multiplicative alignment with the product of the prime factors of n (its radical).
- Forever, because the CRT never forgets.
- Which prevents n from repeating.

- In aggregate these observations establish an information channel that amounts to an "odd prime factor set cursor".
- The process of accumulating coprimality through iteration acts like a sieve over radical lattices, which are coprime by definition.
- The coprimality is propagated forward by the CRT, forcing n to (effectively) enumerate odd prime factor sets.

- Ignoring 2 and 3, the product of the higher prime factors (cursor primes) can be called their "cursor radical", because they implement a walk over radicals formed from just those primes. 
- The cursor radical can be defined R(n)=rad(n) / gcd(rad(n), 2*3). Every odd-to-odd step of the Collatz map we have gcd(R(n), R(n')) = 1. The Collatz map is a strictly monotonic walk down the poset of cursor radicals.
- The +1 operation cycles *radicals*, in natural order.
- Collatz runs in a permutation of natural order (3n), therefore Collatz also cycles cursor radicals. Once a set of cursor prime factors (uniquely represented by its radical) has been seen in Collatz order, it will not be seen again modulo that radical until it cycles.

- Natural ordering modulo any set of primes does not present every combination evenly. Numbers with higher prime factors not represented in the set will alias numbers sharing the same subset of lower factors.
- For example, there are only 16 unique combinations of the factors {2, 3, 5, 7}, but they cycle with period 2x3x5x7 = 210. Inside that cycle, the first 3 also cycle with period 2x3x5 = 30, and the first two with period 2x3 = 6. That 6-cycle repeats 5 times in the 30 cycle, against every phase of prime 5. That 30-cycle repeats 7 times in the 210-cycle, and so on as you add higher primes. This is the pattern of natural ordering, 

- Considering factors {2, 3, 5, 7} the empty set {} appears 48 times in the first 210 integers, representing numbers having none of those factors. This is modeled by Euler's Totient Theorem: φ(N) = φ(2)φ(3)φ(5)φ(7) = 1 * 2 * 4 * 6 = 48 gaps in the cycle.
- During the cycle, the sequence of factor sets appears chaotic, but the order they appear is strict and deterministic: nested prime subcycles.
- The full factor set 2x3x5x7 appears exactly *once* in the cycle of 210, at the end.
- That's the radical for that set of primes, and the radical has that property whatever set of primes you choose: exactly one instance per cycle.

- Each set of odd prime factors above three is unique among odd numbers on a Collatz orbit. That combination of factors will be seen only once per cycle. This is the cursor walk.
- This is independent of factor exponent. Gaining coprimality with 5x7 (for example) also makes n coprime to 5^2x7^3, because they have the same radical.

- Many small values of n share factor sets that differ only in their exponents. These are a "jackpot" of coprimality, because the entire radical equivalence class of composite numbers is sieved out together.
- Small radicals cycle quickly, so it's possible for combinations of low primes to reappear in long orbits. Radicals with higher primes have no chance of return.

- The prime numbers 3 (click) and 2 (bang) are the functional machinery of Collatz. 
- Define the "cursor primes" to be all the *rest* of them, the set of odd primes above 3. Their presence/absence is the persistent state that allows n to walk unique factor sets.
- The +1 operation steps cursor primes through this cycle, unimpeded by the 3n and /2 steps (because they only change n by factors of 3 or 2), making the set of cursor primes Collatz-invariant.
- The 3n and /2 operations are permutations over all prime rings. They hop n to another number, but always with the *same* cursor radical. The +1 step in natural order (nested prime cycles) is made from there. 
- For example, 41 is a prime number. In natural order, it successor is 2x3x7 = 42, and incrementing 41 would normally change the set of cursor primes from {41} to {7}. However, the 3n operation jumps to 41x3 = 123, making the cursor primes the factor set above 3 {41}, then steps in natural order from there, to 124, which has factors 2x2x31, or cursor prime set {31}, which will pass unmodified through the following /2 steps. 
- Collatz then is a permutation of the natural order of succession of cursor prime factor sets. 3n has full coverage over prime rings (it just has to go around them 3 times), the sequence cycles modulo the product of primes, and the +1 operation is applied in shuffled but deterministic order.




- When +1 steps past a radical, the radical goes to the back of that very long cycle. This works the same no matter what set of primes you choose to look at.
- For example, becoming coprime to 35 (the radical of all numbers having factor set {5, 7}) under the +1 operation means it will take 34 more steps to reach another multiple of 35. All numbers with the same factor prime set, like 5x5x5x7, are unreachable by n under Collatz rules until then, at any magnitude.
- We know that at n-35 and n+35, we will not find a number with the factor set {5, 7}, that is, any multiple of 35. That radical (factor set) is *forbidden* at those offsets from n, until a full cycle completes.
- Low magnitude n are "immune" to the presence of higher factors in natural cycle order. Their reduced space of factor sets (reachable radicals) cycles more quickly. Large numbers cycle a large space of potential factors. Small numbers have fewer potential factors. 
- This is the case at every scale. The set of potential factors monotonically increases, but numbers are usually "behind" the maximum for whatever subcycle they are in.

- As Collatz iterates, and coprimality is gained with more cursor radicals, these prohibitions on factorization accumulate, because +1 runs strictly in natural order, and +1 systematically moves each radical encountered to the back of its cycle.


- The odd-to-odd Collatz map is ergodic in 2-adic space, but biased towards lower radicals in integers, because small factors are more common.
- The carry-propagation step in Collatz is a lossless-dissipative (but not reversible) shift into dyadic form, enforced by the finite MSB.
- The presence of the MSB is what allows integer Collatz to make forward progress.

- Non-factor residues appear "scrambled" because they are mutating to maintain existing odd congruences in a constrained way. This is a feature, not a bug.

- Eventual misalignment with all odd prime factor sets under the Collatz map will force n to a power of two.
- Collatz runs in the space of prime residues, which represent CRT constraint state. This determines congruence (or lack thereof). The value of n is indexing this space. It is chaotic-looking but stepping very carefully.

- Because what's evolving is "the set of odd prime factors above three", the actual magnitude of n at any time has little value. It is a pointer to the richer state information in n's local neighborhood.
- The value of n never repeats. This is a clue, because doing that long term requires non-trivial persistent state.
- Long orbits can pass through small values of n and recover. That requires indirection. A handful of bits can't hold enough state to support the complex long term behavior observed; n must represent something more than itself.
- Accumulating modular constraint is the reason n seems to find so many prime numbers: it has no choice. As n becomes coprime with common factor sets, composite numbers become unreachable.

- The orbit of Collatz is an intersection of constraint systems.
- The value of n also sits in a web of modular constraints with its neighbors on the number line.
  - The neighboring evens share no factors.
  - The previous even shares no factors above 3 with the next odd, and vice versa.
  - Neighboring odds share no factors above 5, etc. Factor constraint drops off with distance from n.
- The value of n gains coprimality as it iterates and the neighbors of n must not contradict that relationship, which funnels evolution.


- The local environment E(n) is the complete local state vector of an integer n. It is the set of measurements of n's local neighborhood that are sufficient to uniquely determine n's global arithmetic properties, including its prime factorization. It's a map from a neighborhood of integers to a vector of their 2-adic and p-adic properties.
- Iₙ is the set of integers being measured. Here we focus on n's immediate neighbors, {n-1, n+1}, but this set can include more distant neighbors.
- Pₙ is the set of small "test" primes, e.g., {3, 5, 7, ..., p} where p ≤ log(n)².
- E(n) = {m ↦ (k(m), {oc(m) mod p})_p∈Pₙ | m ∈ Iₙ}
- The minimal environment {n-1, n+1} is a **static snapshot**, and what you need for "spectroscopy", to read the prime factors of n itself. 
- The extended environment {..., n-1, n+1, ...} is more like a movie frame, capturing the local "slope" or "momentum" of the environmental patterns by sampling farther from n.
- The E(n) environment of n = p^k reveals the structure of prime powers within the 2-adic landscape.
- For an odd prime p and any odd exponent k, the 2-adic valuation of p^k-1 matches the base prime's neighbor, that is: v₂(p^k-1) = v₂(p-1). The prime `p` projects a stable 2-adic "shadow" across all its odd powers.
- Odd integer n is divisible by prime p iff oc(n-1) * 2^k(n-1) ≡ -1 mod p, and oc(n+1) * 2^k(n+1) ≡ 1 mod p.
- For any even m, oc(3m) = 3 * oc(m).
- For n ≡ 1 (mod 8), oc(n'-1) = 3 * oc(n-1), and v₂(n'-1) = v₂(n-1) - 2.
- For n ≡ 3 (mod 4), i.e. n ≡ 3, 7 (mod 8), oc(n'+1) = 3 * oc(n+1), and v₂(n'+1) = v₂(n+1) - 1.
- For n ≡ 5 (mod 8), v₂(n-1) = 2, and v₂(3n+1) > 2.
- For any even exponent k, the valuation undergoes a phase shift, and is amplified: v₂(p^k - 1) ≥ 3. 
- The parity of the exponent k in p^k is encoded directly into its environmental fingerprint E(p^k), and we can read it instantly by observing v₂(p^k - 1).
- The LSB of odd n is 1, so n ≡ 1 mod 2. There is also a distinct MSB if n > 1, which it is, letting us say v₂(n-1) + v₂(n+1) ≥ 3.
- For odd prime p and odd k such that n = p^k, v₂(n-1) = v₂(p-1).
- For even integer m > 0, gcd(oc(m), oc(m+2)) = 1, which means the odd cores of n's even neighbors are coprime, and the odd prime factor sets of those odd cores disjoint.
- For any odd integer n > 1, the 2-adic valuations of its neighbors are asymmetric. One neighbor will have k = 1, the other >= 2. This is a consequence of the structure of integers modulo 4. If n=4k+1, then n-1 is a multiple of 4 and n+1 is only a multiple of 2. If n=4k+3, the roles reverse.
- These things together show that the 2-adic landscape around an odd number is intrinsically lopsided at any magnitude. The multiplicative (odd core) and dyadic (v₂) components are governed by rigid rules of **segregation** and **asymmetry**.
- The space of E(n) is a brittle lattice defined by these forced relationships. 

- An odd-to-odd Collatz orbit is a deterministic walk through the residue product space G = ∏ₚ(Z/pZ) over the odd primes dividing n (above 3).
- For each p>3, the odd-to-odd step translates to “n → n+1 (mod p)”. All radical bits (prime factors) become coset coordinates, and Collatz steps advance diagonally in this space; carry shifts, odd core merges, and residuation by 2 are just coordinate transforms.
- Stopping distance for n is exactly the time until the “radical mask” (bit vector of surviving odd primes) is fully “burned”: that is, until you first hit (modulo all p in the radical) n+d ≡ −1 (mod p) for all p.
- This is a simultaneous hitting-time for the tuple (p₁-1, p₂-1,...) steps forward from n in each ring. Chinese Remainder Theorem gives a unique solution d (mod ∏p) to these congruences: the “naive” diagonal cycle. But Collatz merges/hubs cause in-practice much shorter real walks, as radical constraints collapse when the walk encounters a “merge” in the odd core graph.
- Short stopping distances correspond to radicals whose cycles quickly align with −1 in all coordinates, i.e., dense merges. Long stopping distances correspond to radicals made of rare/large primes, forcing long walks before alignment and delayed merges.
- The process of “burning” radicals is visible in the binary/ternary trace as avalanche “triangles”—each big carry-or-resonance event is the moment a prime constraint gets sieved out of play.

- The Collatz map contains infinite sequences of deterministic "almost-doublings" that form an odd-to-odd stopping time equivalence class.
- Using the graph and always following the link with lowest k-value reduces astronomical n to representative "seed" values for the equivalence class in log2 n steps.
- The seed values represent the first appearance of a specific unique odd prime factor set in the integers. The seeds are the set of radicals of primes above 3 (the cursor radicals).
- There is an infinitude of such seeds, because there is an infinitude of prime numbers with unique factor sets (themselves).
- The seeds are are canonical for the entire infinite class of numbers sharing their factor set.
- The Collatz problem reduces to solving stopping distance for these representative seeds: the set of cursor radicals.

- Each odd seed is the base of an infinite 2-adic chain of even multiples of powers of two. These even numbers are an odd-to-odd stopping time equivalence class, because the Collatz /2 rule collapses them all immediately to the same odd seed.
- Seeds partition the set of odd integers into classes: every odd number belongs to exactly one seed’s “doubling chain”.
- Adjacent to every even number on the seed's doubling chain are two odd neighbors. One or both will have the same odd-to-odd stopping time as the seed for this reason.
- Some odd composite numbers work as "hubs" where multiple seed chains join, and the rest of the path is shared.
- The hubs are fed by "connectors" with a similar (but simpler) pattern of doubling, and these funnel into it. The density of connectors increases with n.
- The pattern of self-similar odd core connectivity is unavoidable, because even at astronomical values, the rules of 2-adic valuation (remove power of two factors) connect n regularly to much smaller odd core values. The odd cores have unlimited "reach" in this way.
- Continuing from a seed, using the lowest-k traversal rule, intermediate values are no longer in the same stopping time equivalence class, and paths merge quickly into common "trunk lines". There is no stopping time differentiation below the seed.
- The complete lowest-k root walk across the odd-core connectivity graph (n -> seed -> 1) uniquely identifies n.
- But let's talk about something different.

- The *Goldbach* conjecture is that every even integer 2n above two can be expressed as the sum of two primes p1 and p2. It can be restated this way: for every integer n > 1, there exists an offset d (0 <= d < n) such that both n-d and n+d are prime.
- For this to be true, they must both survive the Sieve of Eratosthenes, meaning for any prime r < sqrt(n+d), neither n-d nor n+d can be divisible by r.
- The process can be seen as a double-sieve, every prime r knocking out values of d that would make either n-d or n+d composite.
- The Goldbach conjecture then is that for any n > 1, there exists an offset d which survives this sieve, and the numbers at both offsets are prime, and sum to 2n.
- The constraints of the Collatz process accumulate the opposite way, with coprimality *preventing* symmetrical radicals at radical offsets, while in Goldbach we're searching for a pre-existing state of coprimality.
- The value of n in Goldbach sits at the intersection of two symmetrical sequences in natural order, one ascending from 1, the other descending from 2n-1, corresponding to pairs of integers n-d (representing p1) and n+d (representing p2). 

- The odd core fingerprint of n-d (p1) is the pair of 2-adic valuations (odd core and k) of p1-1 and p1+1, and the fingerprint on the other side the valuations of p2-1 and p2+1.
- Let f1, f2, f3, f4 represent those 4 fingerprint indices n-d-1, n-d+1, n+d-1, n+d+1. Note the outer two (f1 and f4) sum to 2n, and the inner two (f2 and f3) do as well.
- The space of integer odd cores is self-similar with nested binary topology, and n-d and n+d are at different levels in this hierarchy. The properties of their fingerprints cycle with powers of two, not just with binary cycling, but fractal  and define the landscape that d traverses in two directions while sampling radicals, which are all coprime to two. This is like "grinding" prime numbers against the dyadic hierarchy, exploring all combinations of relationships.
- If the k-values for the outer pair f1 and f4 don't match, the valuation of 2n will be the minimum of them. Likewise the inner pair f2 and f3: 2n's valuation will match the lower of theirs. This is a structural property of the odd core mapping, and applies at any magnitude. 
- For even 2n > 2 and offset d < n, if the fingerprints of n-d and n+d have symmetric k-values (i.e., k(f1) = k(f3) and k(f2) = k(f4)), then n-d and n+d are coprime to each other and to all primes p dividing the difference 2d.

- Euler's totient theorem shows that nested prime cyles are "gappy", and new primes have plenty of opportunity to slip through.

- Natural order cycles radicals in both directions, but these sequences are _not_ symmetrical.
- The odd-core connectivity graph is inherently unbalanced. Half of primes appear on either side, but the spacing of primes varies, so they all have a different local relationship to the tree topology. Visually it looks like a sort of twisting of branches against each other.
- At every step d and n cycle relationships in steps of 1, n-d and n+d cycle relationships in steps of 2, 2n's local neighbors cycle relationships against n-d and n+d independently, it's a grinder of rotating congruences.
- n-d and n+d therefore also cycle relationships in steps of 4, in addition to 2.




\begin{lstlisting}
                                     
                <──────(n-d)──────── n ────────(n+d)──────>
                         │           ║           │                  
                         │           ║     :     │                       :     :     :     :
7 ───■──────■──────■─────│■──────■───║──■──│───■─│────■──────■──────■────│─■───│──■──│───■─│────■────── 7
5 █────■────■────■────■──│─■────█────║────■│───■─│──■────■────█────■────■│───■─│──■──│─■───│█────■────■ 5
3 █──■──█──■──█──■──█──■─│█──■──█──■─║█──■─│█──■─│█──■──█──■──█──■──█──■─│█──■─│█──■─│█──■─│█──■──█──■─ 3
2 █─■─■─█─■─■─█─■─■─█─■─■│█─■─■─█─■─■║█─■─■│█─■─■│█─■─■─█─■─■─█─■─■─█─■─■│█─■─■│█─■─■│█─■─■│█─■─■─█─■─■ 2
                                     ║     │     │                       │     │     │     │           
                                     ║     ^   MATCH                     ^     ^     ^     ^     
    From a given number n, step      ║     │     │                       │     │     │     │           
    the same distance d in both      ║───■─│────■│─────■──────■──────■───│──■──│───■─│────■│─────■───── 7
    directions. The same pattern     ║────█│───■─│──■────■────■────■────█│───■─│──■──│─■───│■────■────█ 5
    of gaps in any prime cycle       ║─■──█│─■──█│─■──█──■──█──■──█──■──█│─■──█│─■──█│─■──█│─■──█──■──█ 3
    will reappear, but backwards,    ║■─■─█│■─■─█│■─■─█─■─■─█─■─■─█─■─■─█│■─■─█│■─■─█│■─■─█│■─■─█─■─■─█ 2
    and offset by a multiple of 2.   ║     :     │                       :     :     :     :
    Some gaps will not match up,     ║           │                       
    but one or more always will.     n ────────(n-d)──────>
                                     ║       BACKWARDS
                                     ║
\end{lstlisting}

- A prime p can only be a potential common factor of both n-d and n+d if p is a prime factor of 2n itself. For any sieving prime r that does not divide 2n, it is structurally impossible for n-d and n+d to be simultaneously divisible by r.
- As d increases, the set of its prime factors rad(d) changes. The primality of n-d requires that rad(n-d) and rad(d) must be disjoint. This imposes a powerful, dynamic constraint: d is forced to be composed of primes that are not currently factors of n-d.
- The simplest implementation of this dual sieve: let D = {1 <= d < n}, and for each prime r <= sqrt(2n), remove from D the d such that n-d ≡ 0 mod r, or n+d ≡ 0 (mod r). By inclusion/exclusion and the CRT, the number of surviving d is at least n ∏_{p ≤ √(2n)} (1 - 2/p). For finite n, this number is positive, meaning there is always a surviving d such that n-d and n+d are coprime to all p≤2n, and therefore themselves prime.







\begin{thebibliography}{9}

\end{thebibliography}

\end{document}



